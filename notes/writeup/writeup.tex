\documentclass[11pt,onecolumn]{IEEEtran}

\renewcommand{\thefootnote}{$\star$}

\usepackage[arrowdel]{physics}
\usepackage{siunitx}
\usepackage{tabularx}
\usepackage{graphicx}
\usepackage{calc}
\usepackage{listings}
\usepackage{cleveref}
\usepackage[margin=0.75in]{geometry}

\author{Henry Mitchell}
\title{LoRa Setup Writeup}
\date{}

\begin{document}
\maketitle

\section{Introduction}
\label{sec:intro}

The purpose of this experiment was to examine the direct wireless communication between two devices with SX1276RF1KAS Transceivers and Arduino Uno boards.
As it stands, no gateway has been set up (though that is in the works), so source code that allows direct communication between two nodes was used.
This experiment implemented an approach using SX1276RF1KAS Transceivers and Arduino Uno boards.
Communication was established between the two units before sensors were attached to the client unit, and the data from those sensors were sent to the server unit.

\section{Materials}
\label{sec:materials}

The following were used in developing this system:
\begin{itemize}
  \item
    SX1276RF1KAS $\times$ 2
    \begin{itemize}
      \item These are the LoRa units which transmit and receive the data from one Arduino to another
    \end{itemize}
  \item
    915MHz Antenna (yellow) $\times$ 2
    \begin{itemize}
      \item These are the antennae which came with the LoRa units, and are plugged into the ``HF'' port on the LoRa units
    \end{itemize}
  \item
    Arduino Uno board $\times$ 2
    \begin{itemize}
      \item These are the micro-controllers which determine which data to collect and send
    \end{itemize}
  \item
    USB 2.0 Type B cable $\times$ 2
    \begin{itemize}
      \item These are to connect the Arduino units to a computer, in order to upload instructions to them
    \end{itemize}
  \item
    Jumper Wires
    \begin{itemize}
      \item These are to connect the Arduino and LoRa units, as well as the Arduino units and sensors
    \end{itemize}

\end{itemize}

The pin connections used in the setup of this system are shown in \cref{tab:pinmap}.
\begin{table}[ht]
  \centering
  \begin{tabular}{c | c | c}
    Purpose      & LoRa & Arduino \\ \hline
    Power supply & 2 (VDD\_RF)   & \SI{3.3}{\volt} \\
    & 22 (VDD\_ANA) & \\
    & 34 (VDD\_FEM) & \\ \hline
    Ground & 32 (GND) & GND \\ \hline
    SPI & 1 (SCK) & D13 \\
    & 3 (MOSI) & D11 \\
    & 8 (MISO) & D12 \\
    & 7 (NSS) & 10 \\ \hline
    Digital I/O & 12 (DIO0) & 2 \\
    & 5 (DIO1) & 6 \\
    & 17 (DIO2) & 7 \\ \hline
    Reset & 10 (NRESET) & 8 \\ \hline
    RXTX & 13 (RXTX) & 3
  \end{tabular}
  \caption{Pin mapping for the experiment.}
  \label{tab:pinmap}
\end{table}

\section{Setup}
\label{sec:setup}

Once the pins are connected as shown in \cref{tab:pinmap}, communication between the two units can start.
The first way to test this system was to upload

\end{document}

%%% Local Variables:
%%% mode: latex
%%% TeX-master: t
%%% End:

